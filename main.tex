\documentclass[12pt]{article}
\usepackage{sbc-template}

\usepackage[utf8]{inputenc}  
\usepackage{import}
\usepackage{indentfirst}
\usepackage{hyperref}
\usepackage{graphicx,url}
\usepackage{bookmark}

\sloppy

\title{RISCV e NanvixOS}

\author{Lucas S. de Oliveira\inst{1}, Thiago H. Nogueira\inst{1}}


\address{Instituto de Informática -- Pontifícia Universidade Católica de Minas Gerais \\
  (PUCMG)
  Belo Horizonte -- MG -- Brazil
  \email{\{lucas.oliveira.1201561, thiago.nogueira\}@sga.pucminas.br}
}

\begin{document}
    \maketitle

    \import{sections}{abstract.tex}
    \import{sections}{resumo.tex}
    \import{sections}{introducao.tex}
    \import{sections}{metodologia.tex}
    \import{sections}{conclusao.tex}

    % Sessão temporária apenas para compilação
    \section{References}

    Bibliographic references must be unambiguous and uniform.  We recommend giving
    the author names references in brackets, e.g. \cite{knuth:84},
    \cite{boulic:91}, and \cite{smith:99}.
    
    The references must be listed using 12 point font size, with 6 points of space
    before each reference. The first line of each reference should not be
    indented, while the subsequent should be indented by 0.5 cm.
    
    \bibliographystyle{sbc}
    \bibliography{sbc-template}
\end{document}