\begin{resumo}
    Sistemas operacionais são desenvolvidos com o intuito de possuir a melhor otimização e funcionamento para diversas arquiteturas de 
    processadores. A implementação desses é um processo delicado, já que diversas tecnologias diferentes emergem constantemente. Neste 
    trabalho apresentamos o desenvolvimento do \textit{port} do sistema operacional Nanvix para uma arquitetura RISC-V, que vai ser emulada a 
    partir de uma ferramenta denominada PULP SDK. Para verificar o funcionamento do \textit{port}, testes automatizados foram realizados, desde o
    nível de kernel ao de usuário. 
\end{resumo}
