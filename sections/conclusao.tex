\section{Conclusão}

Este trabalho teve como foco apresentar um passo a passo de como realizar um \textit{port} funcional para uma arquitetura (RISC-V) no sistema operacional NanvixOS. 
As instruções modificadas podem ser usadas como uma forma de aprendizado e de base para novos pesquisadores. O \textit{port} foi adicionado ao sistema e está disponível para uso no repositório do GitHub.

Pensando em trabalhos futuros, poderia-se usar a ferramenta de \textit{benchmark} que a organização do Nanvix possui para verificar a eficiência do \textit{port} em diferentes arquiteturas e, com isso, pensar em otimizações para o código do sistema. Além disso, com base nos testes, seria possível verificar em qual cenário determinada arquitetura se destaca. Além disso, este \textit{port} ajudará a viabilizar a execução real do Nanvix na plataforma PULP no projeto CEVERO da UFRN.
