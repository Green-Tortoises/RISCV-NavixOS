\section{Introdução}

\emph{NanvixOS} é um sistema operacional educacional que tem o intuito de ensinar pessoas que se interessam em \emph{hack} de \emph{kernel}. 
O sistema foi implementado em x86, uma arquitetura muito difundida atualmente, mas que não é a mais eficiente para sistemas 
embarcados por conta do consumo superior de energia se comparado a outras arquiteturas de processador. O sistema do Nanvix foi 
implementado durante a graduação de um aluno da Pontifícia Universidade Católica de Minas Gerais e ganhou forças quando outras 
universidades começaram a adotá-lo para ensinar sistemas operacionais para seus alunos.

Esse projeto cresceu e agora novos desafios apareceram, o interesse de suportar arquiteturas \emph{RISC-V} apareceu devido ao 
interesse dessa tecnologia no mercado nos últimos anos e o aumento do número de empresas integrando essa tecnologia como a \emph{Nvidia} 
ou a \emph{Western Digital}. Com esse cenário a transição se tornou muito viável, não sendo necessário pagar royalties pelo \emph{Instruction Set Architecture}
(\emph{ISA}), como também possuindo um consumo energético mais eficiente se comparado ao antigo x86.  

Com essa situação em mente, o objetivo do artigo é mostrar os estudos realizados no sistema \emph{NanvixOS}, tanto quanto os 
estudos realizados para o \emph{RISC-V}, para a nova etapa desse projeto.
