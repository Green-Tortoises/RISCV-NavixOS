\subsection{RISC-V}

O RISC-V \cite{RISCV_Org} é um conjunto de instruções (\textit{Instruction Set Architecture} - ISA) baseado em princípios de RISC (\textit{Reduced Instruction Set Computing}). 
Diferentes de outros designs de ISA, RISC-V é de código aberto, o que possibilita o uso dessa arquitetura pela comunidade livre de \textit{royalties}.

    \begin{figure}[h!]
        \centering
        \includegraphics[width=\textwidth]{images/pulp_family.png}
        \caption{Pipeline do RISC-V \\ \cite{PULP-Plataform} } % https://github.com/nanvix/hal/pull/650/files
    \end{figure}

Projetos como o PULP \cite{PULP-Plataform} e também algumas empresas, estão se interessando mais pelo RISC-V. É o caso da Nvidia \cite{Nvidia_Riscv} que 
está utilizando esses chips dentro das placas de vídeos da nova linha Ampere, e a Western Digital \cite{WesternDigital_Riscv}, 
em seus dispositivos de armazenamento. Por conta de sua versatilidade em resolver uma grande gama de problemas diferentes, 
a adoção dessa ISA tem crescido consideravelmente nos últimos anos \cite{RISCV_growth}. A projeção é que até 2025, essa 
arquitetura já movimente em torno de um bilhão de dólares \cite{RISCV_market_growth} e com isso irá apenas crescer e se 
tornar mais simples e acessível.
