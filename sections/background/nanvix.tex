\subsection{Nanvix}

Nanvix é um sistema operacional com origem em uma abordagem educacional \cite{NanvixEdu} e que se baseia na estrutura do Unix System V. Embora o \emph{design} seja simples, o Nanvix evoluiu para uma versão distribuída com compatibilidade com a arquitetura \textit{lightweight manycore} MPPA-256.

\begin{figure}[h!]
    \centering
    \includegraphics[width=\textwidth]{images/overview_nanvix.png}
    \caption{Arquitetura do Nanvix \\ \cite{penna:tel-03545212}}
\end{figure}

A Figura 1 mostra a organização da arquitetura do Nanvix. Sua estrutura é baseada em um layout de quatro camadas: (i) Nível de Hardware; (ii) Kernel; 
(iii) Serviços do Sistema; (iv) Software para Usuário.

Na camada (iii), a HAL é responsável por possibilitar a portabilidade do Nanvix para diversos processadores \emph{lightweight manycores}. Enquanto o \emph{Microkernel} 
oferece recursos para o sistema com apenas a utilização de apenas um cluster, além de rodar em \emph{privileged mode} e possuir um nível de abstração mínimo do SO. 
É possível obter mais informações sobre o Nanvix nos estudos realizados por Pedro Penna \cite{penna:tel-03545212}.
