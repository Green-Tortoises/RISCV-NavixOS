\subsection{Nanvix}

Nanvix é um sistema operacional educacional, que se baseia na estrutura do Unix System V. Seu design é simples e moderno, 
para ajudar pessoas interessadas em sistemas embarcados há aprender sobre hack de kernel. 

\begin{figure}[h!]
    \centering
    \includegraphics[width=\textwidth]{images/nanvix_architecture.png}
    \caption{Arquitetura do Nanvix \\ \cite{ArquiteturaNanvix}} % https://github.com/nanvix/hal/pull/650/files
\end{figure}

A arquitetura do Nanvix é mostrada na Figura 1. Nela é possível observar que a estruturação de duas camadas: (i) O nível da kernel, 
que roda em privileged mode, possui acesso a todo o hardware; (ii) O nível de usuário, que roda em unprivileged mode, onde o software 
não possui acesso direto ao hardware e é responsável pela interação com o usuário. 

A kernel se baseia em uma arquitetura monolítica, sendo estruturada dessa forma em diversos sub-sistemas: (i) camada de abstração de hardware, 
(ii) sistema de gerencia de memoria, (iii) sistema de gerencia de processo e (iv) sistema de arquivos. Para mais informações relacionadas a arquitetura 
do Nanvix, confira esse artigo [citação]