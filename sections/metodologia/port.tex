\subsection{Port do Nanvix para RISC-V}

O problema que existia no sistema era o uso de uma versão do GCC mais atual que implementa pseudo instruções reduzidas que não são suportadas pelo Pulp SDK. 

Para começar será necessário algum sistema operacional compatível para o teste. Seja alguma distro baseada no Debian, como o Ubuntu, ou um Mac, este que 
precisará de algumas adaptações para funcionar corretamente (não há suporte no momento para o Windows). 

Para compilar o sistema é necessário usar o gcc com a flag TARGET definida para “qemu-riscv”, isso configura o makefile para compilar corretamente a 
hal (Hardware Abstraction Layer) e a barelib. É necessário ter um cross-compiler para RISC-V instalado corretamente no computador e os pacotes do qemu 
para emulação de um processador RISC-V. Toda essa informação está disponível na documentação oficial do Nanvix [https://github.com/nanvix/hal/blob/dev/README.md]. 

Com todas as ferramentas em mãos o trabalho é fazer um backport da versão do compilador gcc que estava sendo usado no projeto (versão 1.10 para 1.9.1) As versões 
diferem na forma de escrever as pseudo instruções. Para utilizar essa ferramenta é necessário compilar o cross-compiler da sua plataforma para RISC-V e definir a 
variável de ambiente “PULP\_RISCV\_GCC\_TOOLCHAIN” com o caminho para esse programa. Com isso, será possível executar o pulp e testar as aplicações que já estão 
disponíveis dentro do repositório dele para verificar se a emulação está funcionando corretamente. Sempre é importante ressaltar que é necessário verificar o 
funcionamento da emulação de múltiplas harts ou se está rodando todo o código de forma sequencial. Toda essa informação está disponível na documentação oficial 
do Pulp SDK [https://github.com/pulp-platform/pulp-sdk].

\begin{figure}[h!]
    \centering
    \includegraphics[width=\textwidth]{images/commit_instrucoes.png}
    \caption{Commit com as intruções modificadas \\ \cite{SolucaoInstrucoesDOWNGrade}} % https://github.com/nanvix/hal/pull/650/files
\end{figure}

Para que fosse possível fazer esse backport foi necessário encontrar a documentação do RISC-V em uma versão atual e em uma versão mais antiga, antes de ser suportado 
versões reduzidas de suas instruções. Dentro dos arquivos do Nanvix, três deles  possuem código de máquina dentro do código em C (usando a keyword \_\_asm\_\_ do gcc) 
e que precisariam ser desatualizados. Encontrando todas as referências às instruções reduzidas dentro do repositório, é necessário alterá-las para uma instrução completa
e, com isso, o sistema já estará suportando o Pulp SDK. É possível testar seu funcionamento compilando o Nanvix usando o Pulp.
