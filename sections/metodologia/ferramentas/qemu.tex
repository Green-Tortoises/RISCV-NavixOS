\subsubsection{QEMU}

O qemu [https://www.qemu.org/] é um emulador e um virtualizador de arquiteturas genérico e de código aberto. Esse programa possui a capacidade de executar em \emph{userspace}, ou seja, executar instruções de uma arquitetura de processador em outro processador sem a necessidade de uso de virtualização.

A escolha dessa ferramenta foi essencial no contexto do Nanvix, pois o RISC-V ainda não é um hardware disseminado para que todos possam testar o sistema em hardware real. Com poucas pessoas possuindo um RISC-V em mãos, virtualizá-lo permite que todos possam desenvolver para o Nanvix seja de qual arquitetura possuírem.
