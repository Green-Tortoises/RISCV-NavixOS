\section{Discussão}

Após todas as etapas realizadas, foi necessário verificar o funcionamento das modificações realizadas no código. Para isso foi realizada uma bateria de testes, que 
pode ser organizada em duas etapas: (i) A primeira consiste em testes que são realizados manualmente por outros contribuidores, ao enviar o commit para o repositório 
da organização; (ii) A segunda consiste na realização de testes na plataforma Jenkins, que faz uma compilação do projeto desde o nível de kernel ao nível de usuário. 

É importante citar que os testes no Jenkins não foram realizados de forma correta, já que o servidor está passando por alguns problemas. Com isso, foi necessário que
um contribuidor (João Souto) fizesse os testes manualmente em sua própria máquina para comprovar que as linhas modificadas estavam funcionando conforme o esperado. Com 
essa aprovação, as linhas modificadas foram inseridas no código oficial do sistema disponível em \cite{SolucaoInstrucoesDOWNGrade}, e o backport foi realizado. 