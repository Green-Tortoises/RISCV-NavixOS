\section{Metodologia}

Para começar pode-se acessar a documentação oficial do RISC-V \cite{RISCV_documentation} 
e acessar o projeto do NanvixOS no Github \cite{NanvixOS}. A documentação do 
processador é importante para saber como se executa operações de leitura, escrita e I/O. 
Com essa documentação em mãos pode-se começar a entender como o Nanvix foi implementado para o 
RISC-V, existem outras versões que podem ser usadas como referência para o aprendizado. 
Será necessário também, algum sistema operacional compatível para o teste seja alguma 
distro baseada no Debian, como o Ubuntu ou um Mac, que precisará de algumas adaptações 
para funcionar corretamente (não há suporte no momento para o Windows). 

Começar com uma \emph{good first issue} é uma boa ideia, uma vez que o sistema é composto de muitas linhas de 
código que dificultam o entedimento imediato de todas as funcionalidades do sistema. 
Por exemplo, \emph{Enable SUM Bit} \cite{EnableSUMBit} que já foi corrigida, 
e pode ser usada de documentação para entender como um bug funciona e como solucioná-lo.

\begin{figure}[h!]
    \centering
    \includegraphics[width=\textwidth]{images/enable_sum_bit_issue_github.png}
    \caption{\emph{Bug} do \emph{SUM Bit} não sendo inicializado}
\end{figure}

A correção de um bug como esse não exige mais do que 1 linha de código, mas para alcançar 
essa linha foi necessário estudar como funcionam os bits de permissionamento de execução 
dentro do RISC-V, o que é extremamente proveitoso para novos estudantes. Como disponível na figura 2.

\begin{figure}[h!]
    \centering
    \includegraphics[width=\textwidth]{images/solucao_enable_sum_bit_github.png}
    \caption{Solução para o bug do \emph{SUM Bit} não sendo inicializado \\ \cite{SolucaoEnableSUMBit}}
\end{figure}