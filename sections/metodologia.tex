\section{Metodologia}

É sabido que sistemas operacionais e processadores andam lado a lado. Com isso, ter-se em mãos as especificações do funcionamento de um processador e o conhecimento da gerência de leitura, escrita e I/O é de suma importância. A utilização da documentação oficial do RISC-V [\url{https://riscv.org/technical/specifications/}], como também artigos que realizaram estudos sobre processadores foram de grande relevância para o conhecimento inicial e, consequentemente, o desenvolvimento mais assertivo do projeto. 

Para amadurecer mais no projeto, foi desenvolvido uma série de vídeos com o intuito de familiarização com o sistema que pode ser usado como uma documentação mais focada no boot do Nanvix e que explica como funciona todo o processo de configuração do ambiente, compilação e depuração desse sistema [\url{https://www.youtube.com/playlist?list=PLUzQL-lHV0FjQLqNW_QmxATcYkGfnaOdt}]. 

Depois de um período desenvolvendo os vídeos, se tornou claro que não se estava tendo um resultado promissor. O boot do NanvixOS, como de qualquer outro sistema embarcado, possui uma hierarquia gigantesca arquivos. Por conta disso, era complicado de entender a maior parte do processo e as conclusões obtidas se tornaram rapidamente equivocadas.  

Então começar com uma good first issue é uma boa ideia, uma vez que o sistema é composto de muitas linhas de código. Por exemplo, o Enable SUM Bit [\url{https://github.com/nanvix/hal/issues/357}] solucionado nesse artigo, pode ser usado de documentação para entender como um bug pequeno funciona dentro de todo o sistema e como solucioná-lo.  

\begin{figure}[h!]
    \centering
    \includegraphics[width=\textwidth]{images/enable_sum_bit_issue_github.png}
    \caption{\emph{Bug} do \emph{SUM Bit} não sendo inicializado}
\end{figure}

A correção de um bug como esse não exige mais do que 1 linha de código, mas para alcançar essa linha foi necessário estudar como funcionam os bits de permissionamento de execução no hardware do RISC-V, o que é extremamente proveitoso para novos estudantes. Como visto na figura 2

\begin{figure}[h!]
    \centering
    \includegraphics[width=\textwidth]{images/solucao_enable_sum_bit_github.png}
    \caption{Solução para o bug do \emph{SUM Bit} não sendo inicializado \\ \cite{SolucaoEnableSUMBit}}
\end{figure}

Com uma familiaridade mais notória no ambiente de estudo, devido a correção de bugs, foi possível estabelecer objetivos mais desafiadores. O próximo problema que existia no sistema era o uso de uma versão do GCC mais atual que implementa pseudo instruções reduzidas que não são suportadas pelo PULP SDK. Esse development kit é pensado para emular um ambiente controlado em que são testados instruções de RISC-V o mais próximo do real quanto for possível. 

Para começar será necessário algum sistema operacional compatível para o teste. Seja alguma distro baseada no Debian, como o Ubuntu ou um Mac, este que precisará de algumas adaptações para funcionar corretamente (não há suporte no momento para o Windows). 

Para compilar o sistema é necessário usar o gcc com a flag TARGET definida para “qemu-riscv”, isso configura o makefile para compilar corretamente a hal (Hardware Abstraction Layer) e a barelib. É necessário ter um cross-compiler para RISC-V instalado corretamente no computador e os pacotes do qemu para emulação de um processador RISC-V. Toda essa informação está disponível na documentação oficial do Nanvix [\url{https://github.com/nanvix/hal/blob/dev/README.md}]. 

Com todas as ferramentas em mãos o trabalho é fazer um backport da versão do compilador gcc que estava sendo usado no projeto (versão 1.10 para 1.9.1) As versões diferem na forma escrever as pseudo instruções. Para que o sistema funcione corretamente, foi escolhida uma SDK para o desenvolvimento do Nanvix, o pulp sdk [\url{https://github.com/pulp-platform/pulp-sdk}].
  
Para utilizar essa ferramenta é necessário compilar o cross-compiler da sua plataforma para RISC-V e definir a variável de ambiente "PULP\_RISCV\_GCC\_TOOLCHAIN" com o caminho para esse programa. Com isso, será possível executar o pulp e testar as aplicações que já estão disponíveis dentro do repositório dele para verificar se a emulação está funcionando corretamente, sempre é importante ressaltar que é necessário verificar se está funcionando a emulação de múltiplas harts ou se está rodando todo o código de forma sequencial.

\begin{figure}[h!]
    \centering
    \includegraphics[width=\textwidth]{images/backport_merged.png}
    \caption{Merge do backport funcional \\ \cite{SolucaoEnableSUMBit}} % https://github.com/nanvix/hal/pull/650/files
\end{figure}