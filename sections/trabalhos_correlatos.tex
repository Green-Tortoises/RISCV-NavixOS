\section{Trabalhos Correlatos}

Esta seção apresenta alguns artigos voltados para \textit{port} de sistema operacionais, além de uma visão da própria utilização do Nanvix em outras arquiteturas.

O artigo \emph{Real-Time Operating System for DSP Clusters} \cite{Drozdov2015RealTimeOS} apresenta um modelo de arquitetura e princípios utilizados na construção
do Milandr-OS. Desenvolvido para clusters de multi-processadores com comunicação ponta-a-ponta usando sinais digitais. A criação deste se deu no momento em
que programadores tinham problemas constantes em reescrever algoritmos entre máquinas diferentes, por conta da diferença de arquitetura entre elas. Esse sistema
provê algumas interfaces padronizadas que independente do hardware que for executado, os algoritmos previamente escritos poderão ser reutilizados.

Conforme proposto em \emph{Porting an Operating System on an ARM-based sensor platform} \cite{7372829}, um sistema Linux foi portado para um computador embarcado para
que pudesse ser usado aplicações que se utilizassem de todos os sensores disponíveis na placa. O sistema era originalmente proposto para alguns modelos de processador ARM, mas não era totalmente funcional com os sensores que estavam presentes nesse FPGA e o artigo mencionado conseguiu que todos os sensores da placa funcionassem.

Diversos trabalhos no Nanvix já foram feitos para que outras arquiteturas fossem suportadas. \cite{PENNA20211} que apresenta o uso já funcional de um port do 
Nanvix para outra arquitetura chamada MPPA. Foi usado um processador many-cores de 288 núcleos que mostrou um ganho significativo de desempenho quando usado um 
modelo IKC Facility (modelo esse explicado durante o artigo), quando comparado com o modelo mailbox (esse que era o modelo vigente).

Conforme Pedro Penna, todo o modelo de construção do sistema está contido na tese de doutorado \cite{penna:tel-03545212}. Esse artigo apresenta várias proposições 
de modelos de software construídos para atender com a maior eficiência ao modelo \emph{lightweight manycores}, em que um processador possui núcleos leves (eficientes em energia) e com especialidades diferentes, e que o sistema operacional pode se aproveitar disso, dividindo o trabalho para os núcleos que consigam executar aquela instrução com maior performance e menor gasto energético.

O Nanvix já está sendo projetado para funcionar em ambientes com múltiplos núcleos \cite{MSC-Reinaldo}. Esse, implementou uma biblioteca de paralelismo chamada 
OpenMP dentro do sistema. Com isso, tornou-se possível o uso de múltiplos processadores sendo usados por uma única aplicação para um ganho de desempenho em aplicações 
que podem ser paralelizadas.

Um software de código aberto chamado PULP-SDK foi construído com o intuito de escolher as melhores ferramentas para emulação e virtualização de processadores RISC-V. 
No artigo \emph{GVSoC: A Highly Configurable, Fast and Accurate Full-Platform Simulator for RISC-V based IoT Processors}, foi apresentado uma proposta de software chamada
GVSoC que é totalmente modificável, rápido e preciso na hora de virtualizar esse processador. Por conta disso, foi incluído no PULP e foi utilizado neste artigo, terá
uma seção apenas para citá-lo.

A soma dos esforços de vários contribuidores resultou em um projeto voltado ao uso do Nanvix dentro de um microssatélite que será usado 
para fotografar a floresta Amazônica \cite{RepoCervero}. Esse, chama-se projeto CEVERO (\emph{Chip multi-procEssor for Very Energy-efficient aeRospace missiOns}), e tem 
como membros a Universidade Federal do Rio Grande do Norte (UFRN), a Pontifícia Universidade Católica de Minas Gerais (PUC Minas), que contribui com o SO Nanvix, entre outras instituições. Portanto, cabe ressaltar, que a principal contribuição deste artigo está no \textit{port} do SO Nanvix para a plataforma PULP, que é a base arquitetural do projeto CEVERO.
